%% 
%% Author: papian
%% 26/05/18
%% 

\documentclass{bredelebeamer}

\usepackage[cache=false]{minted}
\usemintedstyle{tango}

%%%%%%%%%% C
\newmintinline{c}{
  fontsize=\small,
  breaklines=true
}

\newminted{c}{
  frame=single,
  framesep=2mm,
  fontsize=\scriptsize,
  mathescape
}

\newminted[ccodeline]{c}{
  frame=single,
  framesep=2mm,
  fontsize=\scriptsize,
  mathescape,
  linenos
}


%%%%%%%%% CMAKE
\newminted{cmake}{
  frame=single,
  framesep=2mm,
  fontsize=\scriptsize,
  mathescape,
  linenos,
  breaklines=true
}

% End minted
%%%%%%%%%%%%%%%%%%%%%%%%%%%%%%%%%%%%%%%%%%%%%%%%%%%%%%%%% 

%%%%%%%%%%%%%%%%%%%%%%%%%%%%%%%%%%%%%%%%%%%%%%%% 

\title[OS Project]{Operating System Project}
% Titre du diaporama

\subtitle{An implementation of server-client database using non-blocking operations}
% Sous-titre optionnel

\author{Julmy, S., Papinutto, M. \& Veillard, S.}
% La commande \inst{...} Permet d'afficher l' affiliation de l'intervenant.
% Si il y a plusieurs intervenants: Marcel Dupont\inst{1}, Roger Durand\inst{2}
% Il suffit alors d'ajouter un autre institut sur le modèle ci-dessous.

\institute[UniFr]{University of Fribourg}

\date{30 Septembre 2018}
% Optionnel. La date, généralement celle du jour de la conférence

\subject{Presentation projet OS Group 4 Sylvain Julmy, Michael Papinutto et Sami Veillard}
% C'est utilisé dans les métadonnes du PDF

\logo{\includegraphics[scale=0.5]{../report/images/unifrlogo.jpg}}

%%%%%%%%%%%%%%%%%%%%%%%%%%%%%%%%%%%%%%%%%%%%%%%%%%%%%%%%%%%%%%%%%%%%% 
\begin{document}

\maketitle

\begin{frame}
  \frametitle{Table of contents}
  \tableofcontents
\end{frame}

\section{Selected Approach}

\begin{frame}
  \frametitle{Multi-threaded Server}
  We selected a multithreaded server as: \\
  \begin{itemize}
  \item Allows to access same data structure
  \item light weight as compared to multi-process server
  \item easier to implement
  \end{itemize}
\end{frame}

\begin{frame}
  \frametitle{To use mutex or not to use mutex}
  Our lock-free implementation allowed us to omit mutex.
  \begin{itemize}
  \item No need to prioritise read or write operations
  \item No deadlock problems
  \item All clients have the same right to access the data structure
  \end{itemize}
\end{frame}

\begin{frame}[fragile]
  \frametitle{Atomic calls}
  Non-blocking operations use atomic operations in order to perform multiple
  operation in one clock cycle. With GCC, we can access them with some
  specific compiler functions
  \footnote{\url{https://gcc.gnu.org/onlinedocs/gcc-4.1.0/gcc/Atomic-Builtins.html}}:
  
  \begin{minted}{c}
    type __sync_fetch_and_add(type *ptr, type value, ...);
    bool __sync_bool_compare_and_swap(
      type* ptr,
      type old_v,
      type new_v,
    ...);
  \end{minted}

  Such operations are in a way an acquire lock - operate - realease lock in
  only one CPU cycle.
\end{frame}

\begin{frame}
  \frametitle{Reversed Split-Ordered Hash-Set}
  
  This implementation\footnote{The Art of multiprocessor programming, Herlihy
    and Shavit} offers a rapid access to the data but might require
  slightly more memory than other data structures.
  
  \begin{itemize}
  \item Buckets are linked in a list which grows automatically when adding
    elements.
  \item To expand the set without to much work, we simply add more shortcut in
    from the first bucket.
  \item We use sentinel between the nodes to avoid "corner case" that occurs
    when deleting a reference by a bucket reference.
  \item The keys are ordered in the reverse binary order : $0x0F101000 \mapsto 0x000808F0$.
  \end{itemize}
  
  \begin{figure}
    \centering
    \includegraphics[width=0.9\textwidth]{../report/images/hashsetFig1.png}
  \end{figure}
\end{frame}

\begin{frame}
  \frametitle{Operation add in this Hash-Set}
  Scheme to add the value 10 in the set :
  \begin{figure}
    \centering
    \includegraphics[width=0.9\textwidth]{../report/images/hashsetFig2.png}
  \end{figure}
\end{frame}

\section{Usage and Command Line Interface}

\begin{frame}
  \frametitle{Program Basic Usage}
  
  \begin{tcolorbox}[taborange,tabularx={X|Y}, boxrule=2pt, title=Server usage]
    \verb+TCP Port+ & 5000 (can be changed in file) \\\hline
    \verb+./server+ & server start
  \end{tcolorbox}
  
  \begin{tcolorbox}[tabvert,tabularx={l|Y}, boxrule=2pt, title=Client Start]
    \verb+./ client <server ip address>+ & client start \\\hline
    \verb+./ client -option <server ip address>+ & client start with options
  \end{tcolorbox}
  
\end{frame}

\begin{frame}
  \frametitle{Client Usage and Commands}
  
  \begin{tcolorbox}[taborange,tabularx={X|l}, boxrule=2pt, title=Client Start with options]
    \verb+-?+ \\ \verb+-h+ \\ \verb+--help+ & client command help \\\hline
    \verb+-f <file>+ \\ \verb+--file <file>+ & client start and execute commands in the file \\\hline
    \verb+-F <file1> ... <fileN>+ \\ \verb+--files <file1> ... <fileN>+ & client start and execute commands in the files
  \end{tcolorbox}
  
  \begin{tcolorbox}[tabvert,tabularx={X|Y}, boxrule=2pt, title=Commands in interactive CLI]
    \verb+add <value> or add <key> <value>+ & add a value to the database  \\\hline
    \verb+ls+ & list content (unordered) \\\hline
    \verb+read_v <key>+ & read value from  key \\\hline
    \verb+read_k <value>+ & read key from value \\\hline
    \verb+rm_v <key>+ & delete value from key \\\hline
    \verb+rm_k <value>+ & delete value from key \\\hline
    \verb+update_kv <value> <newvalue>+ & update an entry
  \end{tcolorbox}
  
\end{frame}

\begin{frame}
  \frametitle{Demo}
  \begin{tcolorbox}
      \begin{center}
        DEMO
      \end{center}
  \end{tcolorbox}
\end{frame}

\section{Tests}

\begin{frame}
  \frametitle{Tests scenarios}
  We tested the following scenarios:
  \begin{itemize}
  \item Scenario with collisions : operations that can collide (a client delete a value before another access it).
  \item Scenario without collision : operations that are ordered so that no collision can occur.
  \item Scenario with many clients : several clients with a similar scenario as no-collisions.
  \end{itemize}
\end{frame}

\begin{frame}
  \frametitle{Collision scenarios}
  $11$ clients and $28$ commands ($308$ operations in total)
  
  \begin{tcolorbox}[tabrouge,tabularx={l|X|X|X}, boxrule=3pt]
    & Add & Read & Delete\\\hline
    Number of errors & 0 & 22 & 0 \\\hline
    Percentage of errors & 0\% & 7.14\% & 0\%
  \end{tcolorbox}
  
\end{frame}

\begin{frame}
  \frametitle{No-collision scenarios}
  $8$ clients and $2700$ commands ($21600$ operations in total)
  
  \begin{tcolorbox}[taborange,tabularx={l|X|X|X}, boxrule=3pt]
    & Add & Read & Delete\\\hline
    Number of errors & 0 & 0 & 0 \\\hline
    Percentage of errors & 0\% & 0\% & 0\%
  \end{tcolorbox}
  
\end{frame}

\begin{frame}
  \frametitle{Many clients scenarios}
  
  $32$ clients and $300$ commands ($9600$ operations in total)
  \begin{tcolorbox}[taborange,tabularx={l|X|X|X}, boxrule=3pt]
    & Add & Read & Delete\\\hline
    Number of errors & 0 & 0 & 0 \\\hline
    Percentage of errors & 0\% & 0\% & 0\%
  \end{tcolorbox}
  This last test required more time than the previous one despite the fact that it has half fewer operations
\end{frame}

\begin{frame}[fragile]
  \frametitle{Encountered problems}
  \begin{itemize}
  \item \verb+strtok+ vs. \verb+strtok_r+
  \item lock-free hashset is hard to understand and to implement
  \end{itemize}
\end{frame}

\section{Conclusion}

\begin{frame}
  \centering
  \textcolor{MidnightBlue}{\Huge Thank's for your attention !}
\end{frame}

\end{document}
